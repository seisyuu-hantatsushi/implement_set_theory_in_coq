\chapter{述語論理}
\section{述語論理}

\subsection{述語とは}
命題論理では,一般化した対象となる変数,これらから新しい対象を作り出す関数,
関係を表す記号が存在せず,表現力が乏しい.
これらの一般化した複数の対象の関係を表す記号を述語と言う.
例として,変数としてい利用される文字,演算記号,関数記号などがある.

\subsection{述語論理}
命題論理を述語を含む論理体系まで拡張する. そのため,記号として以下を導入する.
\begin{enumerate}
 \item \textgt{対象定数}(individual constant) $c_i(i=0,1,2,\dots)$ $c_i$の集合$D$を\textgt{対象領域}という. $D$は空ではない.
 \item \textgt{自由変数}(free variable) $a_i(i=0,1,2,\dots)$ $a,b,c,\cdots$も用いる.
 \item \textgt{束縛変数}(bound variable) $x_i(i=0,1,2,\dots)$ $x,y,z,\cdots$も用いる.
	   2,3は1に対して\textgt{対象変数}と言われる.
 \item \textgt{関数記号}(function symbol) $f_i^n(n=1,2,\dots;i=0,1,2,\dots)$
	   $f_i^n$を$n$変数の関数記号といい, $f_i^n(*_1,\cdots,*_n)$と書くこともある.
	   $n$を$f_i^n$の変数場所(argument place)の個数という.
	   $i$は関数記号が一般に可算個あることを示しているが, 関数記号は全くなくてもよい. 
	   単に$f_i,f^n.f,g$などと省略することもある.
 \item \textgt{述語記号}(predicate symbol) $P_i^n(n=1,2,\dots;i=0,1,2,\dots)$ $P_i^n$を$n$変数の述語記号といい. $P_i^n(*_1,\cdots,*_n)$
 \item \textgt{論理記号}(logical symbol) $\lnot,\land,\lor,\to,\forall,\exists$
\end{enumerate}