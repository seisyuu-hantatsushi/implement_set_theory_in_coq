\chapter{命題論理}
\section{命題論理とは}
命題(proposition)とは
\begin{quotation}
 正しい(真)か正しくない(偽)か決定できる式および文
\end{quotation}
である.\footnote{ゲーテルの不完全性定理はここでは考慮しない.}

論理(logic)とはz
\begin{quotation}
 思考,議論のため法則,ルール
\end{quotation}
である.

ここでは\textgt{論理}は\textgt{数理論理}である.
数理論理は\textgt{概念}を記号に置き換え\textgt{演算}を適用し,形式的に議論を進める.
\textgt{命題論理}(propositional logic)は命題を形式化し演算(法則,ルール)により関係を調べる技法である.

\section{基本記号の導入}
基本記号には,命題定数,命題変数,基本述語,論理記号,括弧がある.
\begin{description}
 \item[命題定数] 真(true),偽(false)のことである. 真は$\top$,偽は$\bot$.
 \item[命題変数] 形式的に,$A,B,C,A_1,B_2,...$と表される.
 \item[基本述語] 命題変数同士での関係を表す記号である. $=,\in$など.
 \item[論理記号] 命題(命題定数と命題変数)に演算を施す. $\lnot,\land,\lor,\to$.
 \item[括弧] 命題同士の基本述語,論理記号での結合に強さを与える.$(,)$.
\end{description}
\newpage

\section{論理式}
命題定数と命題変数を論理記号と括弧で結合したものを,\textgt{論理式}(formula)という.
\begin{dfn}
 論理式
 \begin{enumerate}
  \item 命題定数$\top,\bot$は論理式である.
  \item 命題変数は論理式である.
  \item $A,B$が任意の論理式ならば,$(A) \land (B),(A) \lor (B), \lnot (A), (A) \to (B))$は論理式である.
  \item 1から3を満たすものだけが論理式である.
 \end{enumerate}
\end{dfn}

\section{真理表}
論理式は命題変数が,独立して真や偽を取るとき,真または偽を取ることができる.
論理記号の演算による結果を表にした物を真理表(truth table)と言う.
以下,$\mathfrak{A},\mathfrak{B}$を論理式とする
\begin{enumerate}
 \item 論理積 (logical product, conjunction, $\land$)

   $\mathfrak{A},\mathfrak{B}$がともに真であるとき,$(\mathfrak{A}) \land (\mathfrak{B})$は真である.

	   \begin{tabular}{|c c||c|}
		\hline
		$\mathfrak{A}$    & $\mathfrak{B}$    & $(\mathfrak{A}) \land (\mathfrak{B})$ \\
		\hhline{|==#=|}
		$\top$ & $\top$ & $\top$ \\
		\hline
		$\top$ & $\bot$ & $\bot$ \\
		\hline
		$\bot$ & $\top$ & $\bot$ \\
		\hline
		$\bot$ & $\bot$ & $\bot$ \\
		\hline
	   \end{tabular}

 \item 論理和 (logical sum, disjunction, $\lor$)

	   $\mathfrak{A},\mathfrak{B}$がともに偽であるとき,$(\mathfrak{A}) \lor (\mathfrak{B})$は偽である.

	   \begin{tabular}{|c c||c|}
		\hline
		$\mathfrak{A}$    & $\mathfrak{B}$    & $(\mathfrak{A}) \lor (\mathfrak{B})$ \\
		\hhline{|==#=|}
		$\top$ & $\top$ & $\top$ \\
		\hline
		$\top$ & $\bot$ & $\top$ \\
		\hline
		$\bot$ & $\top$ & $\top$ \\
		\hline
		$\bot$ & $\bot$ & $\bot$ \\
		\hline
	   \end{tabular}

 \item 否定(negation, $\lnot$)

	   $\mathfrak{A}$が真のとき,$\lnot (\mathfrak{A})$は偽である.
	   $\mathfrak{A}$が偽のとき,$\lnot (\mathfrak{A})$は真である.

	   \begin{tabular}{|c||c|}
		\hline
		$\mathfrak{A}$    & $\lnot (\mathfrak{A}) $ \\
		\hhline{|=#=|}
		$\top$ & $\bot$ \\
		\hline
		$\bot$ & $\top$ \\
		\hline
	   \end{tabular}

 \item 含意(implication,$\to$)

	   $\mathfrak{A}$が真で,$\mathfrak{B}$が偽であるときに限り,偽である.

	   含意は$\mathfrak{A}$の仮定より$\mathfrak{B}$が導かれるという内容を形式化した物である.

	   \begin{tabular}{|c c||c|}
		\hline
		$\mathfrak{A}$    & $\mathfrak{B}$    & $(\mathfrak{A}) \to (\mathfrak{B})$ \\
		\hhline{|==#=|}
		$\top$ & $\top$ & $\top$ \\
		\hline
		$\top$ & $\bot$ & $\bot$ \\
		\hline
		$\bot$ & $\top$ & $\top$ \\
		\hline
		$\bot$ & $\bot$ & $\top$ \\
		\hline
	   \end{tabular}
\end{enumerate}

以上の論理記号の結合の強さは,$\lnot > \land = \lor > \to$であり,$\land,\lor$は左結合である.
更に,$((\mathfrak{A}) \to (\mathfrak{B})) \land ((\mathfrak{B}) \to (\mathfrak{A}))$の略号として,
\textgt{同値}(equivalence,$\leftrightarrow$)を導入する.真理表は以下のようになる.

\begin{tabular}{|c c||c|}
 \hline
 $\mathfrak{A}$    & $\mathfrak{B}$    & $(\mathfrak{A}) \to (\mathfrak{B})$ \\
 \hhline{|==#=|}
 $\top$ & $\top$ & $\top$ \\
 \hline
 $\top$ & $\bot$ & $\bot$ \\
 \hline
 $\bot$ & $\top$ & $\bot$ \\
 \hline
 $\bot$ & $\bot$ & $\top$ \\
 \hline
\end{tabular}
\newline
\newline
$\leftrightarrow$は以下を満たす.

\begin{center}
 \begin{tabular}{ll}
  $\mathfrak{A} \leftrightarrow \mathfrak{A}$ & \textgt{反射律} (reflexive law) \\
  $\mathfrak{A} \leftrightarrow \mathfrak{B}$ ならば $\mathfrak{B} \leftrightarrow \mathfrak{A}$ & \textgt{対称律} (symmetric law) \\
  $\mathfrak{A} \leftrightarrow \mathfrak{B}$ かつ $\mathfrak{B} \leftrightarrow \mathfrak{C}$ ならば $\mathfrak{A} \leftrightarrow \mathfrak{C}$ & \textgt{推移律} (transitive law) \\
 \end{tabular}
\end{center}

上の3つの関係を満たす場合,同値関係(equivalence relation)と言う.

論理式内で括弧をつけなくても結合順位が明確な場合,括弧を省略する.

真理表より$\lnot (\mathfrak{A} \land \mathfrak{B}) \leftrightarrow  \lnot \mathfrak{A} \lor \lnot \mathfrak{B}$を演算し真理値を得ることができる.
\begin{center}
 \begin{tabular}{|c c||c|c||c|c|c||c|}
  \hline
  $\mathfrak{A}$ & $\mathfrak{B}$ &
		  $\mathfrak{A} \land \mathfrak{B}$ & $\lnot (\mathfrak{A} \land \mathfrak{B})$ &
				  $\lnot \mathfrak{A}$ & $\lnot \mathfrak{B}$ & $\lnot \mathfrak{A} \lor \lnot \mathfrak{B}$ &
							  $\lnot (\mathfrak{A} \land \mathfrak{B}) \leftrightarrow  \lnot \mathfrak{A} \lor \lnot \mathfrak{B}$ \\
  \hhline{|==#=|=#=|=|=#=|}
  $\top$ & $\top$ & $\top$ & $\bot$ & $\bot$ & $\bot$ & $\bot$ & $\top$ \\
  \hline
  $\top$ & $\bot$ & $\bot$ & $\top$ & $\bot$ & $\top$ & $\top$ & $\top$ \\
  \hline
  $\bot$ & $\top$ & $\bot$ & $\top$ & $\top$ & $\bot$ & $\top$ & $\top$ \\
  \hline
  $\bot$ & $\bot$ & $\bot$ & $\top$ & $\top$ & $\top$ & $\top$ & $\top$ \\
  \hline
 \end{tabular}
\end{center}


\section{恒真}

\begin{thebibliography}{9}
 \bibitem{155129_8Feb20}
		 松本和夫 『復刊 数理論理学』 共立出版 2001
\end{thebibliography}