\chapter{述語論理}
\section{述語論理}

\subsection{述語とは}
命題論理では,一般化した対象となる変数,これらから新しい対象を作り出す関数,
関係を表す記号が存在せず,表現力が乏しい.
これらの一般化した複数の対象の関係を表す記号を述語と言う.
例として,変数としてい利用される文字,演算記号,関数記号などがある.

\subsection{述語論理}
命題論理を述語を含む論理体系まで拡張する. そのため,記号として以下を導入する.
\begin{dfn}
 \label{pred_L}
述語論理の言語
\begin{enumerate}
 \item \textgt{対象定数}(individual constant) $c_i(i=0,1,2,\dots)$ $c_i$の集合$D$を\textgt{対象領域}という. $D$は空ではない.
 \item \textgt{自由変数}(free variable) $a_i(i=0,1,2,\dots)$ $a,b,c,\cdots$も用いる.
 \item \textgt{束縛変数}(bound variable) $x_i(i=0,1,2,\dots)$ $x,y,z,\cdots$も用いる.
	   2,3は1に対して\textgt{対象変数}と言われる.
 \item \textgt{関数記号}(function symbol) $f_i^n(n=1,2,\dots;i=0,1,2,\dots)$
	   $f_i^n$を$n$変数の関数記号といい, $f_i^n(*_1,\cdots,*_n)$と書くこともある.
	   $n$を$f_i^n$の変数場所(argument place)の個数という.
	   $i$は関数記号が一般に可算個あることを示しているが, 関数記号は全くなくてもよい.
	   単に$f_i,f^n.f,g$などと省略することもある.
 \item \textgt{述語記号}(predicate symbol) $P_i^n(n=1,2,\dots;i=0,1,2,\dots)$ $P_i^n$を$n$変数の述語記号といい.
	   $P_i^n(*_1,\cdots,*_n)$と書くこともある.$n$を$P_i^n$の変数場所の個数という.
	   単に,$P_i,P^n,P$などと省略することもある.
 \item \textgt{論理記号}(logical symbol) $\lnot,\land,\lor,\to,\forall,\exists$
\end{enumerate}
\end{dfn}

\begin{dfn}
 \textgt{項}(term)
 \begin{enumerate}
  \item 対象定数は項である.
  \item 自由変数は項である.
  \item $t_1,\dots,t_n$が項であり,$f^n$が$n$変数の関数記号であるならば$f^n(t_1,\dots,t_n)$は項である.
  \item 上の1から3で項とわかるものだけが項である.
 \end{enumerate}
\end{dfn}

自由変数を含まない項を\textgt{閉項}(closed term)と言う.

\begin{dfn}
 \textgt{論理式}(formula)
 \begin{enumerate}
  \item $t_1,\dots,t_n$が項であり,$P^n$が$n$変数の述語記号であるならば$P^n(t_1,\dots,t_n)$は論理式である.
  \item $\phi$が論理式ならば,$\lnot \phi$は論理式である. \newline
		$\phi,\psi$が論理式ならば, $\phi \land \psi, \phi \lor \psi, \phi \to \psi, \phi \leftrightarrow \psi$は論理式である.
  \item $\phi(a)$が論理式で,$a$が自由変数のとき, $\forall x \phi(x)$と$\exists x \phi(x)$は論理式である.
		ただし,$\phi(a)$は論理式$\phi$の中のいくつかの$a$を指定していることを表し,
		$\phi(x)$は$\phi(a)$に現れない任意の束縛変数を$x$としたとき, $\phi(a)$で指定された$a$に$x$を代入したものを表す.
  \item 上の1から3で論理式とわかるものだけが論理式である.
 \end{enumerate}
\end{dfn}

自由変数を含まない論理式を\textgt{閉論理式}(closed formula)という.
論理記号を含まない論理式を\textgt{素論理式}(prime formula)という.

\begin{dfn}
 \textgt{限定作用素}(quantifier)
 \begin{enumerate}
  \item $\forall x$を\textgt{全称作用素}と言う.\\
		$\forall x \phi(x)$は''すべての$x$について,$\phi(x)$が成立する''となる.
  \item $\exists x$を\textgt{存在作用素}と言う. \\
		$\exists x \phi(x)$は''$\phi(x)$を成立するような,$x$が存在する''となる.
  \item $\forall x \phi(x),\exists x \phi(x)$において,
		$\phi(x)$を限定作用素の\textgt{適用範囲}(scope)という.
 \end{enumerate}
\end{dfn}

Def.\ref{pred_L}で行ったことを,\textgt{言語}(language)$L$が定められた言う.
